\documentclass{article}
\usepackage{amsmath}
\usepackage{algorithm}
\usepackage{algpseudocode}
\usepackage{listings, xcolor}
\usepackage{paralist}
\usepackage{chngpage}
\usepackage[colorlinks,linkcolor=black,anchorcolor=black,citecolor=black]{hyperref}

\definecolor{dkgreen}{rgb}{0,0.6,0}
\definecolor{gray}{rgb}{0.5,0.5,0.5}
\definecolor{mauve}{rgb}{0.58,0,0.82}

\lstset{frame=tb,
  language=Java,
  aboveskip=3mm,
  belowskip=3mm,
  showstringspaces=false,
  columns=flexible,
  basicstyle={\small\ttfamily},
  numbers=none,
  numberstyle=\tiny\color{gray},
  keywordstyle=\color{blue},
  commentstyle=\color{dkgreen},
  stringstyle=\color{mauve},
  breaklines=true,
  breakatwhitespace=true
  tabsize=3
}

\title{Project 4 - Distributed Key-Value Store}
\author{
Xie Yuanhang  \\   2011012344\and
Kuang Zhonghong  \\   2011012357\and
Li Qingyang   \\   2011012360 \and
Yin Mingtian   \\   2011012362\and
Wang Qinshi   \\   2012011311}

\changetext{}{+1cm}{-0.5cm}{-0.5cm}{}

\date{}
\begin{document}
\maketitle

\setcounter{section}{-1}
 \section{Repository URL}
\texttt{https://github.com/bcip/KVS}

\section{Task 1}
\subsection{Overview}
\subsection{Correctness Constraints}
\subsection{Declaration}
\subsection{Description}
\subsection{Test}

\section{Task 2}
\subsection{Overview}
Implement registration logic in \texttt{TPCRegistrationHandler}, \texttt{TPCSlaveInfo}, and 
\texttt{TPCMasterHandler.registerWithMaster}.
\subsection{Correctness Constraints}
\begin{compactitem}
	\item Asynchronously handle the socket connections. Use threadpool to ensure that none of its methods are
		blocking.
	\item Maintain infomation about \texttt{SlaveServer} in \texttt{SlaveInfo}.
	\item Register slave server with master. 
\end{compactitem}
\subsection{Declaration}
In this task, we almost don't need extra instance variables or class variables. But for convennience,
we define a regex expression in \texttt{TPCSlaveInfo}that 
\texttt{Pattern SLAVE\_INFO\_REGEX = Pattern.compile(``\^(.*)@(.*):(.*)\$'')}.

\subsection{Description}
\begin{algorithm}
	\caption{\texttt{class TPCRegistrationHandler}}
	\begin{algorithmic}
		\Procedure{handle}{\texttt{Socket} slave}
			\State threadpool.addJob(\texttt{new RegistrationHandler}(slave))
		\EndProcedure
		\Procedure{RegistrationHandler.run}{}
			\State regMsg $\leftarrow$ \texttt{new KVMessage(slave)}
			\State msg $\leftarrow$ \texttt{regMsg.getMessage()}
			\If{!regMsg.getMsgType().equals(``register'')}
				\State ackMsg $\leftarrow$ \texttt{new KVMessage(ERROR\_INVALID\_FORMAT)}
				\State ackMsg.sendMessage(slave)
				\State \Return
			\EndIf
			\State info $\leftarrow$ \texttt{new TPCSlaveInfo(msg)}
			\State master.registerSlave(info)
			\State newmsg $\leftarrow$ ``Successfully registered '' + msg;
			\State ackMsg $\leftarrow$ \texttt{new KVMessage(RESP, newmsg)}
			\State ackMsg.sendMessage(slave)
			\State slave.close()
		\EndProcedure
	\end{algorithmic}
\end{algorithm}
\begin{algorithm}
	\caption{\texttt{class TPCSlaveInfo}}
	\begin{algorithmic}
		\Procedure{TPCSlaveInfo}{String info}
			\State slaveInfoMatcher $\leftarrow$ \texttt{SLAVE\_INFO\_REGEX..matcher(info)}
			\If{!slaveInfoMatcher.matches()}
				\State \texttt{throw new IllegelArgumentException()}
			\EndIf
			\State slaveID $\leftarrow$ \texttt{Long.parseLong(slaveInfoMatcher.group(1))}
			\State hostname $\leftarrow$ \texttt{slaveInfoMatcher.group(2)}
			\State port $\leftarrow$ \texttt{Integer.parseInt(slaveInfoMatcher.group(3))}
		\EndProcedure
		\Procedure{connectHost}{int timeout}
			\State sock $\leftarrow$ \texttt{new Socket()}
			\State sock.setSoTimeOut(timeout)
			\State sock.connect(new InetSocketAddress(hostname, port), TIMEOUT\_MILLISECONDS)
			\State \Return sock
		\EndProcedure
		\Procedure{closeHost}{Socket sock}
			\State sock.close()
		\EndProcedure
	\end{algorithmic}
\end{algorithm}
\begin{algorithm}
	\caption{\texttt{class TPCMasterHandler}}
	\begin{algorithmic}
		\Procedure{registerWiithMaster}{String masterHostname, SocketServer server}
			\State master $\leftarrow$ \texttt{new Socket(masterHostname, 9090)}
			\State regMsg $\leftarrow$ \texttt{new KVMessage(REGISTER, slaveID + ``@'' + server.getHostname()
			+ ``:'' + server.getPort())}
			\State regMsg.sendMessage(master)
			\State respMsg $\leftarrow$ \texttt{new KVMessage(master)}
			\State master.close()
		\EndProcedure
	\end{algorithmic}
\end{algorithm}
\subsection{Test}
\begin{compactitem}
	\item Test whether slave server can register successfully;
	\item Test whether slave infomation are recorded correctly;
	\item Test whether master can register slave successfully in \texttt{TPCMasterHandler}.
\end{compactitem}

\section{Task 3}
\subsection{Overview}
\subsection{Correctness Constraints}
\subsection{Declaration}
\subsection{Description}
\begin{algorithm}
	\caption{Task 3: \texttt{TPCMaster}}
	\begin{algorithmic}
		\Procedure{registerSlave}{TPCSlaveInfo slave}
			\If{size of \texttt{slaves} reach \texttt{numSlaves}}
				\State return
			\EndIf
			\If{map \texttt{slaves} exists \texttt{slave}'s ID}
				\State update the ID at the map with \texttt{slave}
			\Else
				\State add ID and \texttt{slave} to map \texttt{slaves}
			\EndIf
		\EndProcedure
		\Procedure{findFirstReplica}{String key}
			\State hash key as \texttt{hashedKey}
			\State in \texttt{slaves} celling \texttt{hashedKey}
			\If{null}
				\State celling \texttt{slaves} with \texttt{Long} min value
			\EndIf
			\State return the celling value
		\EndProcedure
		\Procedure{findSuccessor}{TPCSlaveInfo firstReplica}
			\State in \texttt{slaves} higher \texttt{firstReplica}'s ID
			\If{null}
				\State celling \texttt{slaves} with \texttt{Long} min value
			\EndIf
			\State return the celling value
		\EndProcedure
	\end{algorithmic}
\end{algorithm}
\begin{algorithm}
	\caption{Task 3: \texttt{TPCClientHandler}}
	\begin{algorithmic}
		\Procedure{TPCClientHandler}{TPCMaster tpcMaster, int connections}
			\State \texttt{this.tpcMaster} \leftarrow \texttt{tpcMaster}
			\State \texttt{this.threadPool} \leftarrow \texttt{new ThreadPool(connections)}
		\EndProcedure
		\Procedure{handle}{Socket client}
			\State add job \texttt{ClientHandler(client)} to \texttt{threadPool}
		\EndProcedure
		\Procedure{run}{()}
			\State get \texttt{msg} from \texttt{client}
			\If{\texttt{msg}'s type is GET}
				\State \texttt{tpcMaster} handleGet with \texttt{msg}
				\State set the return message with value returned
				\State send the message to client
			\ElseIf{\texttt{msg}'s type is PUT}
				\State \texttt{tpcMaster} handleTPCRequest with \texttt{msg} and boolean true
			\ElseIf{\texttt{msg}'s type is DEL}
				\State \texttt{tpcMaster} handleTPCRequest with \texttt{msg} and boolean false
			\EndIf
		\EndProcedure
	\end{algorithmic}
\end{algorithm}
\subsection{Test}

\section{Task 4}
\subsection{Overview}
\subsection{Correctness Constraints}
\subsection{Declaration}
\subsection{Description}
\subsection{Test}

\section{Task 5}
\subsection{Overview}
\subsection{Correctness Constraints}
\subsection{Declaration}
\subsection{Description}
\subsection{Test}
\end{document}
